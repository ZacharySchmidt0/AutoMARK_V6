% This is the Template LATEX file used to create the Automark PDFS
% Feel free to modify as you see fit.
%
% MATLAB runs a find and replace routine when producing latex files
% 
% It looks for variables enclosed by two percent symbols
%
% Full list of variables
%
%	 	- Students Name
%	*SHEETIMAGES* 	- Where all the sheet image code is inserted (using incgraph)
%	*ANNOTATIONTEXT* 	- Pastes all of the text from annotations, Possibly useful?
%	486			- Score
%   627		- Maximum Score
%   Prof. David S. Nobes		- Prof name
%   Mec E 468 Numerical Simulation in Mechanical Engineering Design		- Name of course
%   Win 2022			- Current semester
%   Assignment No 1	- Current assignment
%	14-Apr-2021 14:14:01	-date of drawing creation
%	06-Jul-2022 15:43:36	-date of drawing last save
%	12-Apr-2011 09:02:16	-date of model creation
%	06-Jul-2022 15:39:57		-date of model last save


\documentclass{article}

\usepackage{incgraph}
\usepackage{tikz}
\usepackage{graphicx}
\usepackage{xcolor}
\usepackage{dblfloatfix}
%\newlength{\savedpageheight}
%\newlength{\savedpagewidth}

	\addtolength{\oddsidemargin}{-1.0in}
	\addtolength{\evensidemargin}{-1.0in}
	\addtolength{\textwidth}{1.75in}

	\addtolength{\topmargin}{-1.25in}
	\addtolength{\textheight}{1.75in}

% These are arbitrary TIKZ Commands, It will be pasted into the overlay!
% Read up the 880 page tikz manual if you want to do some fancy things.
% If you don't want an overlay, do not delete these and instead make the command an empty one!
\newcommand{\keySheetOverlay}{\node[above left, text=red] at (16.375in,2.358in) {\huge Assignment Solution} ;}
\newcommand{\studentSheetOverlay}{\node[above left, text=red] at (16.375in,2.358in) {\huge Student Submission} ;}
\newcommand{\studentmarkedSheetOverlay}{\node[above left, text=red] at (16.375in,2.358in) {\huge Automark of Student Submission} ;}
\DeclareRobustCommand{\greencheck}{
  \tikz\fill[scale=0.4, color=0, 212, 28]
  (0,.35) -- (.25,0) -- (1,.7) -- (.25,.15) -- cycle;
}
\begin{document}
	\pagenumbering{gobble}
	%\maketitle
	%  Letterhead of the department
\begin{center}
\includegraphics[scale=0.5]{./IncludeImages/LetterHead.jpg}
\end{center}

%----------------------------------------------------------------------------------------
%	HEADING SECTIONS
%----------------------------------------------------------------------------------------
%\textsc{\LARGE University of Alberta}\\[1.0 cm] % Name of your university/college
%\textsc{\Large MecE 265 Engineering Graphics and CAD}\\[0.5cm] % Major heading such as course name
\begin{center}
\textbf{\huge Mec E 468 Numerical Simulation in Mechanical Engineering Design}\\[1cm] % Major heading such as course name
\textsc{\LARGE AutoMARK REPORT}\\[0.5cm] % Minor heading such as course title
\end{center}
%Adding some space
 \bigskip
 %----------------------------------------------------------------------------------------
%	AUTHOR SECTION
%----------------------------------------------------------------------------------------
\linespread{1.5} % linspacing
\begin{flushleft} \large
\textbf{Semester: Win 2022}\\
\textbf{Instructor: Prof. David S. Nobes}\\
\textbf{Student Name: }\\
\textbf{Assignment: Assignment No 1}\\ 
PROCESSING DATE: {\large \today}\\[0.5cm] % Date, change the \today to a set date if you want to be precise
%{\large \DTMnow}\\[2cm] % current data nad time ; works with datetime2 package
\end{flushleft}
%
%----------------------------------------------------------------------------------------
%	SCORE SECTION
%----------------------------------------------------------------------------------------
\begin{center}
 \large \emph {{AUTOMark Assessment Grade:}}
 \large \textbf{486 out of 627}\\[0.5 cm] %
 \large \emph {{AUTOMark Recommended Grade:}}
 \large \textbf{78 out of 100}\\[0.5 cm] %
\end{center}
\vspace{1.0\baselineskip} % Adding desired space
\begin{flushleft}
\textbf{NOTE: This grade is preliminary only and needs to be confirmed.}\\[1.0 cm] %
\end{flushleft}
%----------------------------------------------------------------------------------------
%	Other Notes SECTION
%----------------------------------------------------------------------------------------
The following pages include each of the drawings in the following order:
\begin{itemize}
\itemsep0em %setting items spacing
	\item Your submission 
	\item Your submission marked by AutoMARK
	\item The solution
\end{itemize}
%
\textbf{Other important points:}
\begin{itemize}
\itemsep0em %setting items spacing
	\item Examples are given on eClass of how to interpret the mark-up symbols used by AUTOMark.
	\item If you have any questions, discuss with you TA in the next lab time.
\end{itemize}

\begin{flushleft} \large
\\[0.5cm]
DRAWING CREATION DATE: {14-Apr-2021 14:14:01}\\
DRAWING LAST SAVE DATE: {06-Jul-2022 15:43:36}\\
MODEL CREATION DATE: {12-Apr-2011 09:02:16}\\
MODEL LAST SAVE DATE: {06-Jul-2022 15:39:57}\\
\end{flushleft}

\pagebreak
 \begin{samepage}
\addtolength{\topmargin}{-.25in}

\linespread{1.5}
\begin{center}
\begin{table*}[!b]
\centering
	\begin{tabular}{|c|c|}
	 \large \emph  Symbol/Colour &  \large \emph  Meaning \\ 
		{\greencheck} & No deductions on feature  \\ 
		\textcolor[RGB]{255, 0, 0 }{Colour }& Incorrect value \\ 
		\textcolor[RGB]{0, 0, 0 }{Colour } & Miscellaneous error \\ 
		\textcolor[RGB]{0, 0, 255 }{Colour } & Incorrect Position \\ 
		\textcolor[RGB]{237, 177, 32 }{Colour } & Unrecognized feature \\ 
		\textcolor[RGB]{255, 0, 0 }{Colour } & Missing feature \\ 
		\textcolor{yellow }{?}& Feature not found on key\\  
\end{tabular}
\end{table}
\end{center}
 \large \emph AutoMARK details:
\begin{itemize}
\itemsep0em %setting items spacing
	\item Sheetnames should contain only alphabetical characters 
	\item Weights of feature properties are set by the marker
	\item AutoMARK v 4.0 software written and designed by Owen Stadlwieser
\end{itemize}
 \large \emph AutoMARK Criterion (The weights of these criterion are decided by the marker):
\begin{itemize}
\itemsep0em %setting items spacing
	\item DRAWING: SheetOrder, ExtraSheets
	\item SHEET: SheetPaperSize, SheetScale, SheetTemplate, SheetExtraBOMS, SheetExtraViews, SheetViewTypes, SheetIntersectingBallons
	\item BILLOFMATERIALS: BOMTableType, BOMNumberColumns, BOMNumberRows, BOMPosition, BOMTableHeight, BOMTableWidth, BOMFontType, BOMFontSize
	\item VIEW: ViewScale, ViewDisplayStyle, ViewExtraDimension, ViewPosition, ViewExtraCenterMarks, ViewMass, ViewMaterial, ViewExtraDatums, ViewWrongProjection, ViewExtraCenterlines
	\item DIMENSION: DimensionDangling, DimensionWrongView, DimensionPosition, DimensionArrowSide, DimensionValue, DimensionBadText
	\item CENTERLINE: CenterlineDangling, CenterlinePosition
	\item CENTERMARK: CentermarkDangling, CentermarkPosition, CentermarkShowlines, CentermarkAngle, CentermarkConnectionLines, CentermarkExtensions, CentermarkGap, CentermarkSize, CentermarkGroupedCorrectly
	\item DATUM: DatumDangling, DatumWrongView, DatumPosition, DatumLabel, DatumDisplayStyle, DatumFilledTriangle
	\item Ballon: BallonDangling, BallonPosition
\end{itemize}
 \end{samepage}
	\incgraph[overlay={\studentSheetOverlay}]{"./Sheets/SOLUTION".png}
\incgraph[overlay={\studentmarkedSheetOverlay}]{"./ReportImages/SOLUTION".png}
\incgraph[overlay={\keySheetOverlay}]{"./KeySheets/SOLUTION".png}

\end{document}