% This is the Template LATEX file used to create the Automark PDFS
% Feel free to modify as you see fit.
%
% MATLAB runs a find and replace routine when producing latex files
% 
% It looks for variables enclosed by two percent symbols
%
% Full list of variables
%
%	STUDENTNAME* 	- Students Name
%	SHEETIMAGES* 	- Where all the sheet image code is inserted (using incgraph)
%	ANNOTATIONTEXT* 	- Pastes all of the text from annotations, Possibly useful?
%	SCORE*			- Score
%       SCOREOUTOF*		- Maximum Score
% 	ASSIGNMENTNO*		- The Assignment Number

\documentclass{article}

\usepackage{incgraph}
\usepackage{tikz}
\usepackage{graphicx}

%\newlength{\savedpageheight}
%\newlength{\savedpagewidth}

%------------------------------
% PAGE SIZE /LAYOUT
%------------------------------
	\addtolength{\oddsidemargin}{-1.0in}
	\addtolength{\evensidemargin}{-1.0in}
	\addtolength{\textwidth}{1.75in}

	\addtolength{\topmargin}{-1.25in}
	\addtolength{\textheight}{1.75in}

% These are arbitrary TIKZ Commands, It will be pasted into the overlay!
% Read up the 880 page tikz manual if you want to do some fancy things.
% If you don't want an overlay, do not delete these and instead make the command an empty one!
\newcommand{\keySheetOverlay}{\node[above left, text=red] at (16.375in,2.358in) {\huge Assignment Solution} ;}
\newcommand{\studentSheetOverlay}{\node[above left, text=red] at (16.375in,2.358in) {\huge Student Submission} ;}
\newcommand{\studentmarkedSheetOverlay}{\node[above left, text=red] at (16.375in,2.358in) {\huge Automark of Student Submission} ;}

\title{AUTOMARK REPORT}
\date{}

\begin{document}
	\pagenumbering{gobble}
	%\maketitle
	%  Letterhead of the department
	
\begin{center}
\includegraphics[scale=0.5]{./IncludeImages/LetterHead.jpg}
\end{center}

%----------------------------------------------------------------------------------------
%	HEADING SECTIONS
%----------------------------------------------------------------------------------------
%\textsc{\LARGE University of Alberta}\\[1.0 cm] % Name of your university/college
%\textsc{\Large MecE 265 Engineering Graphics and CAD}\\[0.5cm] % Major heading such as course name
\begin{center}
\textbf{\huge  Mec E 265 Engineering Graphics and CAD}\\[1cm] % Major heading such as course name
\textsc{\LARGE AUTOMark REPORT}\\[0.5cm] % Minor heading such as course title
\end{center}
%Adding some space
 \bigskip
 %----------------------------------------------------------------------------------------
%	AUTHOR SECTION
%----------------------------------------------------------------------------------------
\linespread{1.5} % linspacing
\begin{flushleft} \large
\textbf{Semester: Winter 2021}\\
\textbf{Instructor: Prof. David S. Nobes}\\
\textbf{Student Name: *STUDENTNAME*}\\
% Manually add the Assignment number HERE
\textbf{Assignment: Assignment No.1}\\ 
PROCESSING DATE: {\large \today}\\[0.5cm] % Date, change the \today to a set date if you want to be precise
%{\large \DTMnow}\\[2cm] % current data nad time ; works with datetime2 package
\end{flushleft}
%
%----------------------------------------------------------------------------------------
%	SCORE SECTION
%----------------------------------------------------------------------------------------
\begin{center}
 \large \emph {{AUTOMark Assessment Grade:}}
 \large \textbf{*SCORE* out of *SCOREOUTOF*}\\[0.5 cm] %
\end{center}
\vspace{1.0\baselineskip} % Adding desired space
\begin{flushleft}
\textbf{NOTE: This grade is preliminary only and needs to be confirmed.}\\[1.0 cm] %
\end{flushleft}
%----------------------------------------------------------------------------------------
%	Other Notes SECTION
%----------------------------------------------------------------------------------------
The following pages include each of the drawings in the following order:
\begin{itemize}
\itemsep0em %setting items spacing
	\item Your submission 
	\item Your submission marked by AutoMARK
	\item The solution
\end{itemize}
%
\textbf{Other important points:}

\begin{itemize}
\itemsep0em %setting items spacing
	\item Examples are given on eClass of how to interpret the mark-up symbols used by AUTOMark.
	\item If you have any questions, discuss with you TA in the next lab time.
\end{itemize}


*SHEETIMAGES*

\end{document}